\documentclass[ 12pt, a4paper]{article}
% Use the option doublespacing or reviewcopy to obtain double line spacing
% \documentclass[doublespacing]{elsart}

\usepackage[utf8]{inputenc}
\usepackage[backend = biber, maxcitenames=2,uniquelist=minyear]{biblatex}

\AtEveryBibitem{\clearfield{number}}
\AtEveryBibitem{\clearfield{doi}}
\AtEveryBibitem{\clearfield{url}}
\AtEveryBibitem{\clearfield{issn}}
\AtEveryBibitem{\clearfield{isbn}}

\addbibresource{references.bib}

\usepackage{color,graphicx,tikz}
\usetikzlibrary{positioning,arrows}
% The amssymb package provides various useful mathematical symbols
\usepackage{mathtools,amssymb,amsmath,mathdots}
\usepackage[mathscr]{eucal} %just for the font \mathscr
\usepackage{setspace}
\usepackage{hyperref}

\usepackage{tikz}

\renewcommand{\vec}[1]{\boldsymbol{#1}}
\renewcommand{\thefootnote}{\fnsymbol{footnote}}

\newcommand{\inc}{\mathrm{inc}}
\newcommand{\scat}{\mathrm{s}}

\newcommand{\ii}{\mathrm{i}}
\newcommand{\ee}{\mathrm{e}}

\renewcommand{\vec}[1]{\boldsymbol{#1}}


\begin{document}

\title{Multiple scattering of waves}
\author{
Artur L. Gower$^{a}$,\\
\footnotesize{$^{a}$ School of Mathematics, University of Manchester, Oxford Road, Manchester, M13 9PL,UK}
}
\date{\today}
\maketitle

\begin{abstract}
Here we show and deduce the T-matrix and a general multiple scattering formulation which can be adapted to acoustics, electromagnetism, and elasticity. For details on each specific physical medium see the other documents.
\end{abstract}

\noindent
{\textit{Keywords:} Multiple scattering, T-matrix, Scattering matrix}

\section{Using a T-matrix}
A T-matrix denotes how one single particle scatters waves~\parencite{ganesh_far-field_2010,ganesh_algorithm_2017}.

For convenience and generality we denote:
\begin{equation}
\begin{aligned}
    & \mathrm u_{n}(k\rv) = \text{outgoing spherical waves},
    \label{eqn:outgoing_waves_and_regular_waves}
    \\
    & \mathrm v_{n}(k\rv)= \text{regular spherical waves},
 \end{aligned}
\end{equation}
where $n$ denotes a multi index which depends on the dimension and if the waves are scalar or vector fields. For example, for scalar waves in three spatial dimensions we have 
\begin{equation}
  \begin{aligned}
      & \mathrm u_{n}(k\rv) = \mathrm h^{(1)}_\ell(k r) \mathrm Y_n(\rvh),
      \\
      & \mathrm v_{n}(k\rv)= \mathrm j_\ell(k r) \mathrm Y_n(\rvh),
   \end{aligned}
  \end{equation}
  where $n$ is a multi index $n=(\ell,m)$, with $\ell=0,1,2,3\ldots$ and $m=-\ell,-\ell+1,\ldots,-1,0,1,\ldots,\ell$. Here $\mathrm h_\ell^{(1)}(z)$ and $\mathrm{j}_\ell(z)$ are the spherical Hankel and Bessel functions respectively, and $\mathrm Y_n$ are the spherical harmonic basis functions that are orthonormal with respect to the standard inner product on the unit sphere~\cite{Edmonds1974}.

Any incident wave and scattered wave\footnote{For the scattered wave we need only use outgoing spherical waves when measuring the field outside of a sphere which completely encompasses the particle.}, centred at the same coordinate axis, can be written as
\begin{align}
  \label{eqn:incident regular expansion}
  & \ui = \sum_{n} g_n \mathrm v_{n}(k\rv),
  \\
  & \us = \sum_{n} f_n \mathrm u_{n}(k\rv),
\end{align}
for vector waves, such as elastic waves, $f_n$ and $\mathrm u_{n}(k\rv)$ are both vectors for each $n$, with $f_n \mathrm u_{n}(k\rv)$ being the inner product between these two vectors, the same is true for $g_n$ and $\mathrm v_{n}(k\rv)$.  

The T-matrix is an infinite matrix such that
\begin{equation}
  f_n = \sum_{n'} T_{nn'} g_{n'},
\end{equation}
where for vector waves $T_{nn'}$ is a matrix multiplied with the vector $g_{n'}$.
Such a matrix $T$ exists when scattering is a linear operation (elastic scattering).

\subsection{Interal field}
We can also estimate the field inside the particle by assuming that the field is smooth and continuous. This approach is generally not true for vector wave equations, but is exact for homogeneous spheres and cylinders, but not for a \href{acoustics.pdf}{Circular cylindrical capsule}.

Assume the field inside the particle can be described by a regular spherical series:
\begin{equation}
  \vi = \sum_n b_n \mathrm v_n(k_o \rv),
\end{equation}
where $k_o$ if the particles wavenumber. Now if we assume that the total field is continuous everywhere so that $\ui + \us = \vi$ on the boundary of the particle. If the field was smooth enough, we could analytically extend the field $\vi$ to a spherical boundary, with radius $a$, which contains the particle. Let's take this as an assumption and equate $\ui + \us = \vi$ for $r=a$. Due to orthogonality of the angular components of the basis functions this will result in
\begin{equation}
   g_n \mathrm v_{n}(k\rv) + f_n \mathrm u_{n}(k\rv) = b_n \mathrm v_n(k_o \rv), \quad \text{for} \;\; |\rv| = a
\end{equation}
using the T-matrix we can then write $g_n = T_{nm}^{-1} f_m$, which substituted above leads to
\begin{equation}
    b_n = \frac{1}{\mathrm v_n(k_o \rv)}[ \mathrm v_{n}(k\rv) T_{nm}^{-1} f_m + \mathrm u_{n}(k\rv) f_n], \quad \text{for} \;\; |\rv| = a.
\end{equation}

\section{General multiple scattering}

For multiple scattering in higher dimensions and for vector wave equations we use the notation given in \cite{gower2020effective}.

For a point $\rv$, outside of the circumscribed spheres of all particles, we can write the total field $u(\rv)$ as a sum of the incident wave $\ui(\rv)$ and all scattered waves in the form~\cite{Kristensson2015a,Kristensson2016,Linton+Martin2006}
\begin{equation} \label{eqn:total_discrete_wave}
    u(\rv) = \ui(\rv) + \us(\rv), \quad \us(\rv) =  \sum_{i=1}^N \sum_n f_n^i \mathrm u_n (k \rv - k \rv_i),
\end{equation}
where we assumed $ |\rv - \rv_i| > a_i $ for $i=1,2,\ldots N$, the $f_n^i$ are coefficients we need to determine. 
% where again:
% \begin{equation}
% \left\{\begin{aligned}
%     & \mathrm u_{n}(k\rv) = \text{outgoing spherical waves},
%     \label{eqn:outgoing_waves_and_regular_waves}
%     \\
%     & \mathrm v_{n}(k\rv)= \text{regular spherical waves},
%  \end{aligned}\right.
% \end{equation}
% where $n$ denotes a multi index which depends on the dimension and if the waves are scalar or vector fields.

In general, we can write the multiple scattering system in the form:
\begin{equation}\label{eqn:multiple-scattering-general}
   \alpha_n^i = g_{n}^i
    +\sum_{\substack{j=1\\j\neq i}}^N \sum_{n' n''}\mathcal{U}_{n''n}(k\rv_i - k\rv_j)T_{n''n'}^j \alpha_{n'}^j,
\end{equation}
for $i=1,2,\ldots,N$, where $f_n^i = \sum_{n'} T^i_{nn'}\alpha_{n'}^i$ and $\mathcal{U}_{nn'}$ is a translation matrix \cite{Bostrom+Kristensson+Strom1991,Friedman+Russek1954}. Let $\rv'=\rv+\dv$, then
the translation matrices for a translation $\dv$ can be defined by the property~\cite{Bostrom+Kristensson+Strom1991}
  \begin{equation}\label{eq:translation_spherical_waves}
 \left\{\begin{aligned}
   &\mathrm{v}_n(k\rv')=\sum_{n'}\mathcal{V}_{nn'}(k\dv)\mathrm{v}_{n'}(k\rv),\quad \text{ for all }\dv
   \\
 &\mathrm{u}_n(k\rv')=\sum_{n'}\mathcal{V}_{nn'}(k\dv)\mathrm{u}_{n'}(k\rv),\quad |\rv|>|\dv|
 \\
   &\mathrm{u}_n(k\rv')=\sum_{n'}\mathcal{U}_{nn'}(k\dv)\mathrm{v}_{n'}(k\rv),\quad |\rv|<|\dv|
   \\
 \end{aligned}\right.
 \end{equation}

The coefficients $g_{n'}^i$ depend on the form of the incident wave. If we can use the representation~\eqref{eqn:incident regular expansion} then we have that  
\[
  g_{n}^i = \sum_{n'}\mathcal V_{n' n}(\rv_i) g_{n'}.
\]

\subsection{Turning equations into code}
For easy implementation we need the functions:
\[
\psi_\inc \mapsto g_n^j \quad \text{and} \quad \text{particle} \mapsto T_{nn'}^j.
\]

For efficient implementation we rewrite~\eqref{eqn:multiple-scattering-general} as a matrix equation. Let
\begin{align}
  &(\vec \alpha_j)_n =  \alpha_n^j, \quad (\vec g_j)_n =  g_n^j,
  \\
  &(\vec T_j)_{nn'} = T_{nn'}^j, \quad (\vec {\mathcal U}_{j \ell})_{n'n} = \mathcal U_{n'n}(k \rv_j - k\rv_\ell),
  % H_{n'-n}(k R_{\ell j})\ee^{\ii(n'-n)\Theta_{\ell j}},
\end{align}

 Then
\begin{equation}
 \sum_{\ell}(\delta_{j \ell} +  (\delta_{j \ell}-1) \vec {\mathcal U}_{j \ell}^{\mathrm T} \vec T_\ell) \vec \alpha_\ell  =  \vec g_j,
\end{equation}
where $\cdot ^{\mathrm T}$ is the transpose operation. The above then leads to a block matrix equation:
\begin{equation}
  \begin{bmatrix}
    \vec I & - \vec {\mathcal U}_{1 2}^{\mathrm T}\vec T_2 & \cdots & - \vec {\mathcal U}_{1 (N-1)}^{\mathrm T} \vec T_{N-1} & - \vec {\mathcal U}_{1 N}^{\mathrm T} \vec T_N \\
    - \vec {\mathcal U}_{2 1}^{\mathrm T} \vec T_1 & \vec I &  - \vec {\mathcal U}_{2 3}^{\mathrm T} \vec T_3 & \cdots & - \vec {\mathcal U}_{2 N}^{\mathrm T} \vec T_N \\
     & \vdots & & & \vdots \\
     - \vec {\mathcal U}_{N 1}^{\mathrm T} \vec T_1  & \cdots & \cdots & -  \vec {\mathcal U}_{N (N-1)}^{\mathrm T} \vec T_{N-1} & \vec I
  \end{bmatrix}
  \begin{bmatrix}
    \vec \alpha_1 \\
    \vec \alpha_2 \\
    \vdots \\
    \vec \alpha_N
  \end{bmatrix}
   = \begin{bmatrix}
     \vec g_1 \\
     \vdots \\
     \vec g_N
   \end{bmatrix}
\end{equation}

\section{Periodic multiple scattering}

Here we consider a unit cell filled with particles that is repeated periodically. The particles can take any positions within the cell.

Let us start with the simplest case of identical particles centered at the positions $\vec r_1 \in \mathcal P$, where $\mathcal P$ is some countable set of vectors we define later.
% where $m \in \mathbbZ$, $p = 1, 2, \ldots, P$, and $\vec v_p \in \mathbb R^d$.

The total field is again given by~\eqref{eqn:total_discrete_wave}. However, if we assume the source is periodic with
\begin{equation}\label{eqn:incident-periodic}
  u_\mathrm{inc}(\vec r) = u_\mathrm{inc}(\vec r + \vec r_1), \quad \text{for every} \;\; \vec r_1 \in \mathcal P,
\end{equation}
then, due to symmetry, the scattering coefficients of every particles is the same $f_n := f^i_n$, and as a result the total field is given by
\[
u(\vec r) = u_\mathrm{inc}(\vec r) + \sum_n f_n \sum_{i} \mathrm u_n(k \vec r - k \vec r_i).
\]

Taking $\vec r = \vec v + \vec r_j$, we can then write the wave arriving at (or exciting) the particle at $\vec r_j$ in the form
\[
u^{j}_\mathrm{ex}(\vec v) = u_\mathrm{inc}(\vec v) + \sum_n f_n \sum_{i \not = j} \mathrm u_n(k \vec v + k \vec r_j - k \vec r_i),
\]
where we used~\eqref{eqn:incident-periodic}. Now we assume that $\vec v$ is close to the boundary of particle in the unit cell (which is needed to apply boundary conditions), so that $|\vec v | < |\vec r_j -  \vec r_i|$ for $j \not = i$, which allows us to use~\eqref{eq:translation_spherical_waves}${}_3$ to write the above as a series of regular spherical waves centred at $\vec r_j$ in the form
\[
u^{j}_\mathrm{ex}(\vec v) = \sum_{n_1} g_{n_1} \mathrm v_{n_1}(\vec r) + \sum_n f_n \sum_{i \not = j} \sum_{n_1} \mathcal U_{n n_1}( k \vec r_j - k \vec r_i) \mathrm v_{n_1}(k \vec v).
\]
Using the T-matrix formulation, we can now link the scattering coefficients $f_n$ to the coefficients of the regular wave above to get
\begin{equation}
  f_{n'} =  \sum_{n_1} T_{n' n_1} g_{n_1} + \sum_{n n_1} f_n \sum_{i \not = j} T_{n' n_1} \mathcal U_{n n_1}( k \vec r_j - k \vec r_i),
\end{equation}
which can be solved for $f_n$. The main issue is how to truncate the series $ \sum_j \mathcal U_{n n_1}( k \vec r_j)$ in $j$, but I think this would work quite well.

\printbibliography

\end{document}

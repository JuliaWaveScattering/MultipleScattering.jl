\documentclass[ 12pt, a4paper]{article}
% Use the option doublespacing or reviewcopy to obtain double line spacing
% \documentclass[doublespacing]{elsart}

\usepackage[utf8]{inputenc}
\usepackage[backend = biber, maxcitenames=2,uniquelist=minyear]{biblatex}

\AtEveryBibitem{\clearfield{number}}
\AtEveryBibitem{\clearfield{doi}}
\AtEveryBibitem{\clearfield{url}}
\AtEveryBibitem{\clearfield{issn}}
\AtEveryBibitem{\clearfield{isbn}}

\addbibresource{references.bib}

\usepackage{color,graphicx,tikz}
\usetikzlibrary{positioning,arrows}
% The amssymb package provides various useful mathematical symbols
\usepackage{mathtools,amssymb,amsmath,mathdots}
\usepackage[mathscr]{eucal} %just for the font \mathscr
\usepackage{setspace}
\usepackage{hyperref}

\usepackage{tikz}

\renewcommand{\vec}[1]{\boldsymbol{#1}}
\renewcommand{\thefootnote}{\fnsymbol{footnote}}

\newcommand{\inc}{\mathrm{inc}}
\newcommand{\scat}{\mathrm{s}}

\newcommand{\ii}{\mathrm{i}}
\newcommand{\ee}{\mathrm{e}}

\renewcommand{\vec}[1]{\boldsymbol{#1}}


\begin{document}

\title{Multiple scattering of waves}
\author{
Artur L. Gower$^{a}$,\\
\footnotesize{$^{a}$ School of Mathematics, University of Manchester, Oxford Road, Manchester, M13 9PL,UK}
}
\date{\today}
\maketitle

\begin{abstract}
Here we show and deduce the T-matrix and a general multiple scattering formulation which can be adapted to acoustics, electromagnetism, and elasticity. For details on each specific physical medium see the other documents.
\end{abstract}

\noindent
{\textit{Keywords:} Multiple scattering, T-matrix, Scattering matrix}

\section{Using a T-matrix}
A T-matrix denotes how one single particle scatters waves~\parencite{ganesh_far-field_2010,ganesh_algorithm_2017}.

For convenience and generality we denote:
\begin{equation}
\begin{aligned}
    & \mathrm u_{n}(k\rv) = \text{outgoing spherical waves},
    \label{eqn:outgoing_waves_and_regular_waves}
    \\
    & \mathrm v_{n}(k\rv)= \text{regular spherical waves},
 \end{aligned}
\end{equation}
where $n$ denotes a multi index which depends on the dimension and if the waves are scalar or vector fields.

Any incident wave and scattered wave\footnote{For the scattered wave we need only use outgoing spherical waves when measuring the field outside of a sphere which completely encompasses the particle.}, centred at the same coordinate axis, can be written as
\begin{align}
  & \ui = \sum_{n} g_n \mathrm v_{n}(k\rv),
  \\
  & \us = \sum_{n} f_n \mathrm u_{n}(k\rv).
\end{align}
The T-matrix is an infinite matrix such that
\begin{equation}
  f_n = \sum_{n'} T_{nn'} g_{n'}.
\end{equation}
Such a matrix $T$ exists when scattering is a linear operation (elastic scattering).

We can also estimate the field inside the particle by assuming that the field is smooth and continuous. This approximation is exact for homogeneous spheres and cylinders, but not for a \href{acoustics.pdf}{Circular cylindrical capsule}.

Assume the field inside the particle can be described by a regular spherical series:
\begin{equation}
  \vi = \sum_n b_n \mathrm v_n(k_o \rv),
\end{equation}
where $k_o$ if the particles wavenumber. Now if we assume that the total field is continuous everywhere so that $\ui + \us = \vi$ on the boundary of the particle. If the field was smooth enough, we could analytically extend the field $\vi$ to a spherical boundary, with radius $a$, which contains the particle. Let's take this as an assumption and equate $\ui + \us = \vi$ for $r=a$. Due to orthogonality of the angular components of the basis functions this will result in
\begin{equation}
   g_n \mathrm v_{n}(k\rv) + f_n \mathrm u_{n}(k\rv) = b_n \mathrm v_n(k_o \rv), \quad \text{for} \;\; |\rv| = a
\end{equation}
using the T-matrix we can then write $g_n = T_{nm}^{-1} f_m$, which substituted above leads to
\begin{equation}
    b_n = \frac{1}{\mathrm v_n(k_o \rv)}[ \mathrm v_{n}(k\rv) T_{nm}^{-1} f_m + \mathrm u_{n}(k\rv) f_n], \quad \text{for} \;\; |\rv| = a.
\end{equation}

\section{General multiple scattering}

For multiple scattering in higher dimensions and for vector wave equations we use the notation given in \cite{gower2020effective}.

For a point $\rv$, outside of the circumscribed spheres of all particles, we can write the total field $u(\rv)$ as a sum of the incident wave $\ui(\rv)$ and all scattered waves in the form~\cite{Kristensson2015a,Kristensson2016,Linton+Martin2006}
\begin{equation} \label{eqn:total_discrete_wave}
    u(\rv) = \ui(\rv) + \us(\rv), \quad \us(\rv) =  \sum_{i=1}^N \sum_n f_n^i \mathrm u_n (k \rv - k \rv_i),
\end{equation}
where we assumed $ |\rv - \rv_i| > a_i $ for $i=1,2,\ldots N$, the $f_n^i$ are coefficients we need to determine, where again:
\begin{equation}
\left\{\begin{aligned}
    & \mathrm u_{n}(k\rv) = \text{outgoing spherical waves},
    \label{eqn:outgoing_waves_and_regular_waves}
    \\
    & \mathrm v_{n}(k\rv)= \text{regular spherical waves},
 \end{aligned}\right.
\end{equation}
where $n$ denotes a multi index which depends on the dimension and if the waves are scalar or vector fields.

In general, we can write the multiple scattering system in the form:
\begin{equation}\label{eqn:multiple-scattering-general}
   \alpha_n^i=g_{n}^i
    +\sum_{\substack{j=1\\j\neq i}}^N \sum_{n' n''}\mathcal{U}_{n''n}(k\rv_i - k\rv_j)T_{n''n'}^j \alpha_{n'}^j,
\end{equation}
for $i=1,2,\ldots,N$, where $f_n^i = \sum_{n'} T^i_{nn'}\alpha_{n'}^i$ and $\mathcal{U}_{nn'}$ is a translation matrix \cite{Bostrom+Kristensson+Strom1991,Friedman+Russek1954}. Let $\rv'=\rv+\dv$, then
the translation matrices for a translation $\dv$ can be defined by the property~\cite{Bostrom+Kristensson+Strom1991}
  \begin{equation}\label{eq:translation_spherical_waves}
 \left\{\begin{aligned}
   &\mathrm{v}_n(k\rv')=\sum_{n'}\mathcal{V}_{nn'}(k\dv)\mathrm{v}_{n'}(k\rv),\quad \text{ for all }\dv
   \\
 &\mathrm{u}_n(k\rv')=\sum_{n'}\mathcal{V}_{nn'}(k\dv)\mathrm{u}_{n'}(k\rv),\quad |\rv|>|\dv|
 \\
   &\mathrm{u}_n(k\rv')=\sum_{n'}\mathcal{U}_{nn'}(k\dv)\mathrm{v}_{n'}(k\rv),\quad |\rv|<|\dv|
   \\
 \end{aligned}\right.
 \end{equation}

\subsection{Turning equations into code}
For easy implementation we need the functions:
\[
\psi_\inc \mapsto g_n^j \quad \text{and} \quad \text{particle} \mapsto T_{nn'}^j.
\]

For efficient implementation we rewrite~\eqref{eqn:multiple-scattering-general} as a matrix equation. Let
\begin{align}
  &(\vec \alpha_j)_n =  \alpha_n^j, \quad (\vec g_j)_n =  g_n^j,
  \\
  &(\vec T_j)_{nn'} = T_{nn'}^j, \quad (\vec {\mathcal U}_{j \ell})_{n'n} = \mathcal U_{n'n}(k \rv_j - k\rv_\ell),
  % H_{n'-n}(k R_{\ell j})\ee^{\ii(n'-n)\Theta_{\ell j}},
\end{align}

 Then
\begin{equation}
 \sum_{\ell}(\delta_{j \ell} +  (\delta_{j \ell}-1) \vec {\mathcal U}_{j \ell}^{\mathrm T} \vec T_\ell) \vec \alpha_\ell  =  \vec g_j,
\end{equation}
where $\cdot ^{\mathrm T}$ is the transpose operation. The above then leads to a block matrix equation:
\begin{equation}
  \begin{bmatrix}
    \vec I & - \vec {\mathcal U}_{1 2}^{\mathrm T}\vec T_2 & \cdots & - \vec {\mathcal U}_{1 (N-1)}^{\mathrm T} \vec T_{N-1} & - \vec {\mathcal U}_{1 N}^{\mathrm T} \vec T_N \\
    - \vec {\mathcal U}_{2 1}^{\mathrm T} \vec T_1 & \vec I &  - \vec {\mathcal U}_{2 3}^{\mathrm T} \vec T_3 & \cdots & - \vec {\mathcal U}_{2 N}^{\mathrm T} \vec T_N \\
     & \vdots & & & \vdots \\
     - \vec {\mathcal U}_{N 1}^{\mathrm T} \vec T_1  & \cdots & \cdots & -  \vec {\mathcal U}_{N (N-1)}^{\mathrm T} \vec T_{N-1} & \vec I
  \end{bmatrix}
  \begin{bmatrix}
    \vec \alpha_1 \\
    \vec \alpha_2 \\
    \vdots \\
    \vec \alpha_N
  \end{bmatrix}
   = \begin{bmatrix}
     \vec g_1 \\
     \vdots \\
     \vec g_N
   \end{bmatrix}
\end{equation}

\section{Periodic multiple scattering}

Here we consider a unit cell filled with particles that is repeated periodically. The particles can take any positions within the cell.

Let us start with the simplest case of just one particle centred at $\vec r_1$. We assume there are identical particles centered at the positions $\vec r_1 \in \mathcal P$.
% where $m \in \mathbb Z$, $p = 1, 2, \ldots, P$, and $\vec v_p \in \mathbb R^d$.

The total field is again given by~\eqref{eqn:total_discrete_wave}, with $\vec r_i \in \mathcal P$. However, if we assume the source is periodic with
\begin{equation}\label{eqn:incident-periodic}
  u_\mathrm{inc}(\vec r) = u_\mathrm{inc}(\vec r + \vec r_1), \quad \text{for every} \;\; \vec r_1 \in \mathcal P,
\end{equation}
then, due to symmetry, the scattering coefficients are the same $f_n := f^i_n$, and as a result the total field is given by
\[
u(\vec r) = u_\mathrm{inc}(\vec r) + \sum_n f_n \sum_{m p } \mathrm u_n(k \vec r - k \vec r_1 - k m \vec v_p).
\]

Taking $\vec r = \vec v + \vec r_1 + m_1 \vec v_{p_1}$, we can then write the wave arriving at (or exciting) the particle at $\vec r_1 + m_1 \vec v_{p_1}$ in the form
\[
u^{m_1 p_1}_\mathrm{ex}(\vec v) = u_\mathrm{inc}(\vec v + \vec r_1) + \sum_n f_n \sum_{m \not = m_1, p \not = p_1} \mathrm u_n(k \vec v + k m_1 \vec v_{p_1} - k m \vec v_p),
\]
where we used~\eqref{eqn:incident-periodic}. Writing the above as a series of regular spherical waves cented at  then leads to
\[
u^{m_1 p_1}_\mathrm{ex}(\vec v) = \sum_{n_1} g_{n_1} \mathrm v_{n_1}(\vec r) + \sum_n f_n \sum_{m \not = m_1, p \not = p_1} \sum_{n_1} \mathcal U_{n n_1}() \mathrm u_n(k \vec v + k m_1 \vec v_{p_1} - k m \vec v_p),
\]
\printbibliography

\end{document}
